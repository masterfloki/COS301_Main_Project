% project execution for flowchart project -> I suggest we discuss this first before adding any sections etc.

\subsection{Development Methodology}
\begin{flushleft}
%\subsection{Development Methodology}
Our team will be using the \textbf{agile software development methodology}. The agile manifesto states that we find value in the following four points:\\
\begin{itemize}
	\item Individuals and interactions over processes and tools.
	\item Working software over comprehensive documentation.
	\item Customer collaboration over contract negotiation.
	\item Responding to change over following a plan.
\end{itemize}
Although we find value in the items on the right, we value the items on the left even more.\\
In agile development, the testing phase, which is usually separate from development in other methodologies such as Waterfall, runs concurrently with development. This increases the \emph{quality} of the product and means that at any point in the development phase there will be a \emph{working model} available with new features and functionality being added only after they pass vigorous testing.\\
The agile methodology has a proven low project schedule risk and the ability to respond to change, from the client or development team, quickly.
\end{flushleft}

\subsection{Informing the Client}
Our team will have an active, regularly updated, wiki on GitHub and we will also use the GitHub issue tracker with Milestones to keep track of work and goals. We will meet with the client on a weekly or bi-weekly basis to get feedback and discuss progress and problems.

\subsection{Initial Ideas on Solving Some Technical Challenges}
\begin{itemize}
	\item We will use the strategy design pattern to create different classes for the different types of blocks available.
	\item Each block-class will have an execute function that performs the required action.
	\item The flowchart will be implemented as a directed graph with each block holding a reference to its successor(s)
	\item The program will use the Model-View-Controller architectural pattern to separate the interface and the functionality and to simplify flow-chart error checking.
	\item If required, flowcharts will be as stored xml and the program will be able to load these xml files.
\end{itemize}

\subsection{Technologies We will Use}
We plan to use Java with JavaFX to build this application since all our team members know Java and some of us already have experience with JavaFX. It is also platform independent so it can be run on both Windows and Linux.

We will use XML to save flowcharts so a student can save or submit their work. The reason for this choice is that XML is platform independent and can easily be parsed. 

\subsection{What Will The Client Receive}
On completion of the development cycle the client will receive the following:
\begin{itemize}
\item The functional application
\item The program source code
\item Code documentation
\item Requirement and design documentation
\item A user manual for the program.
\end{itemize}