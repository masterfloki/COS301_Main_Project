% project execution for COSBAS -> I suggest we discuss this first before adding any sections etc.
\subsection{Development Methodology}
\begin{flushleft}
%\subsection{Development Methodology}
Our team will be using the \textbf{agile software development methodology}. The agile manifesto states that we find value in the following four points:\\
\begin{itemize}
	\item Individuals and interactions over processes and tools.
	\item Working software over comprehensive documentation.
	\item Customer collaboration over contract negotiation.
	\item Responding to change over following a plan.
\end{itemize}
Although we find value in the items on the right, we value the items on the left even more.\\
In agile development, the testing phase, which is usually separate from development in other methodologies such as Waterfall, runs concurrently with development. This increases the \emph{quality} of the product and means that at any point in the development phase there will be a \emph{working model} available with new features and functionality being added only after they pass vigorous testing.\\
The agile methodology has a proven low project schedule risk and the ability to respond to change, from the client or development team, quickly.
\end{flushleft}

\subsection{Informing the Client}
Our team will have an active, regularly updated, wiki on GitHub and we will also use the GitHub Issue Tracker with Milestones to inform the client on the status of the product, features integrated into the working model and any other information deemed important by the team. We will also ensure to attend an annual two-weekly meeting with the client to give feedback on the product as well as getting new suggestions from the client on possible changes and improvements.

\subsection{Initial Ideas on Solving Some Technical Challenges}
With this project we will be focusing on facial recognition as an authentication service, although we will be developing the product in a highly pluggable manner such that other features like fingerprint scanning will also be supported. The use of third-party API’s will be used for extracting valuable information from images to enable facial recognition.

Multiple quotes for hardware needed will be gathered, primarily from UP vendors, to get the most cost-effective and high quality hardware for the system. A proposed budget and cost of project will be given to the client for approval after all quotes have been scrutinised.

\subsection{Technologies We Aim To Use}
A Java-based server will be used in conjunction with the OpenCV library interface. This decision was made based on the availability of third-party API’s and the ability to attach security certificates.
A highly responsive web-base client-side JavaScript application will also be used as it is not OS dependent, looks and feels like an independent application and can work on both mobile and desktop environments. For calendar functionality mentioned in the functional requirements, we aim to use the Google Calendar API for a more mainstream and accessible system.

To capture images to process for authentication purposes, we aim to use a small lightweight camera that is linked to a Raspberry PI (or similar lightweight computer). This will simply capture the image and send the information to the server for facial recognition and authentication.

\subsection{What Will The Client Receive}
On completion of the development cycle the client will receive the following:
\begin{itemize}
	\item The complete source code.
	\item The complete installation scripts.
	\item An in-depth user manual.
	\item An in-depth installation manual.
	\item A small working prototype system to showcase the system.
	\item All hardware purchased by the client for the system.
\end{itemize}